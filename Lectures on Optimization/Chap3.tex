\chapter{Nonsmooth Convex Optimization}\label{chap:nonsmooth_convex_opt}

\newpage
\section{General Convex Functions}\label{sec:general-convex-funcs}
\subsection{Motivation and Definitions}
In this Chapter we mainly focus on general convex minimization problem
\begin{equation}\label{eq:general-convex-problem}
    \begin{aligned}
        &\min_{\xB \in Q} f_0(\xB) ,\\
        &\text{s.t.}~ f_i(\xB) \le 0,~i = 1, \dots, m,
    \end{aligned}
\end{equation}
where \(Q \subset \R^n\) is a \emph{closed} convex set and \(f_i(\cdot)~(i=0,\dots,m)\) are
\emph{general convex functions}, which means these functions can be \emph{nondifferentiable}.

Note that nonsmooth minimization problems arise frequently in different applications. 
\begin{enumerate}
    \item {
        Quite often, some components of a model are composed of max-type functions.
        \[
            f(\xB) = \max_{1 \le j \le p} f_j (\xB),  
        \]
        where \(f_j(\cdot)\) are convex and differentiable, but it is reasonable to treat max-type as a
        general convex function.
    }
    \item {
        Another source of nondifferentiable functions is the situation when some components of the problem~\ref{eq:general-convex-problem}
        are given \emph{implicitly}.
    }
\end{enumerate}
    
\begin{defn}
    Denote by 
    \[
        \dom f = \{\xB \in \R^n \mid \abs{f(\xB)} \le \infty\}
    \]
    the \emph{domain} of function \(f\). We always assume that \(\dom f \ne \emptyset\).
\end{defn}

\begin{defn}[Convex]\label{defn:general-convex}
    A function \(f(\cdot)\) is called \emph{convex} if if its domain is convex and for all \(\xB, \yB \in \dom f\)
    and \(\alpha \in [0, 1]\) the following inequality holds:
    \begin{equation}
        f(\alpha \xB + (1 - \alpha) \yB) \le \alpha f(\xB) + (1 - \alpha) f(\yB).
    \end{equation}
    If this inequality is strict, the function is called \emph{strictly convex}. We call \(f\) \emph{concave} if \(-f\)
    is convex.
\end{defn}

\begin{lemma}[Jensen's Inequality]\label{lemma:jensen-inequality}
    For any \(\xB_1, \dots, \xB_m \in \dom f\) and positive coefficients such that
    \[
      \sum_{i=1}^{m} \alpha_i = 1,  
    \]
    we have
    \begin{equation}
        f \left( \sum_{i=1}^{m} \alpha_i \xB_i \right) \le \sum_{i=1}^{m} \alpha_i f(\xB_i)
    \end{equation}
\end{lemma}

\begin{proof}[of Lemma.~\ref{lemma:jensen-inequality}]
    pass
\end{proof}

Let us mention \(2\) important consequences of Jensen's inequality.
\begin{coro}\label{coro:jensen-coro-1}
    Let \(\xB\) be a convex combination of points \(\xB_1, \dots, \xB_m\). Then
    \[
        f(\xB) \le \max_{1 \le i \le m} f(\xB_i)  
    \]
\end{coro}
\begin{proof}[of Corollary.~\ref{coro:jensen-coro-1}]
    \[
        f(\xB) = f\left( \sum_{i = 1}^{m} \alpha_i \xB_i \right) 
        \le \sum_{i=1}^m \alpha_i f(\xB_i) \le \max_{1 \le i \le m} f(\xB_i).
    \]
\end{proof}

\begin{coro}\label{coro:jensen-coro-2}
    Let 
    \[
        \Delta = \Conv{\xB_1, \dots ,\xB_m} \equiv 
        \left\{ \xB = \sum_{i=1}^m \alpha_i \xB_i \mid 
        \alpha_i \ge 0, \sum_{i=1}^m \alpha_i = 1 \right\}
    \]
    Then \(\max_{\xB \in \Delta} f(\xB) = \max_{1 \le i \le n} f(\xB_i)\).
\end{coro}

There exist two other equivalent definitions of convex functions.
\begin{thm}\label{thm:convex-defn-1}
    A function \(f\) is convex if and only if for all \(\xB, \yB \in \dom f\) and \(\beta \ge 0\) such that
    \(\yB + \beta (\yB - \xB) \in \dom f\), we have
    \begin{equation}
        f(\yB + \beta (\yB - \xB)) \ge f(\yB) + \beta (f(\yB) - f(\xB))
    \end{equation}
\end{thm}

\begin{thm}\label{thm:convex-epi}
    A function $f$ is convex if and only if its \emph{epigraph}
    \[
        \epi(f) = \left\{ (\xB, t) \in \dom f \times \R \mid t \ge f(\xB) \right\}  
    \]
    is convex set.
\end{thm}
\begin{proof}[of Theorem.~\ref{thm:convex-epi}]
    pass
\end{proof}

\begin{thm}\label{thm:convex-level-sets}
    If a function $f$ is convex, then all level sets
    \[
        \mathscr{L}_f(\beta) = \left\{ \xB \in \dom f \mid f(\xB) \le \beta \right\}
    \]
    are either convex or empty.
\end{thm}
\begin{proof}[of Theorem.~\ref{thm:convex-level-sets}]
    pass
\end{proof}

\begin{defn}\label{defn:closed-func}
    A function \(f\) is called \emph{closed} and convex on a convex set \(Q \subset \dom f\) if its 
    \emph{constrained epigraph}
    \[
        \epi_Q(f) = \{ (x, t) \in Q \times \R \mid t \ge f(\xB) \}.  
    \]
    is a closed convex set. If \(Q = \dom f\), we call \(f\) a \emph{closed convex function}.
\end{defn}

\begin{note}{Note}
    The set \(Q\) is not necessarily closed.
\end{note}

\begin{lemma}
    Let a function $f$ be closed and convex on \(Q\). Then for any closed convex set \(Q_1 \subset Q\), this function is closed and convex on \(Q_1\).
\end{lemma}

\begin{thm}[Topological Properties of Closed Convex Functions]
    Let a function \(f\) be closed and convex.
    \begin{enumerate}
        \item {
            For any sequence \(\{\xB_k\} \subset \dom f\) \emph{convergent} to a point \(\solution \in \dom f\) we have
            \[
                \lim_{k \to \infty}\inf   f(\xB_k) \ge f(\solution).
            \]
            (This means that \(f\) is lower \emph{semi-continuous}.)
        }
        \item {
            For any sequence \(\{\xB_k\} \subset \dom f\) \emph{convergent} to some point \(\solution \notin \dom f\) we have
            \[
                \lim_{k \to \infty} f(\xB_k) = +\infty.  
            \]
        } 
        \item {
            All level sets of the function f are either empty or closed and convex.
        }
        \item {
            Let \(f\) be closed and convex on a set \(Q\) and its constrained level sets bounded. Then problem
            \[
                \min_{\xB \in Q} f(\xB)  
            \]
            is solvable.
        }
        \item {
            Let \(f\) be closed and convex on \(Q\). If the optimal set \(X^* = \arg\min_{\xB \in Q}f(\xB)\) is nonempty and bounded,
            then all level sets of the function \(f\) on \(Q\) are either empty or bounded.
        }
    \end{enumerate}
\end{thm}

\subsection{Operations with Convex Functions}\label{thm:operations-convex}
\begin{thm}
    Let functions \(f_1\) and \(f_2\) be closed and convex on convex sets \(Q_1\) and \(Q_2\), and \(\beta \ge 0\).
    Then all functions below are closed and convex on the corresponding sets \(Q\):
    \begin{enumerate}
        \item \(f(\xB) = \beta f_1 (\xB),~Q = Q_1\).
        \item \(f(\xB) = f_1(\xB) + f_2(\xB),~Q = Q_1 \cap Q_2\).
        \item \(f(\xB) = \max\{f_1(\xB), f_2(\xB)\},~Q = Q_1 \cap Q_2\).
    \end{enumerate}
\end{thm}

\begin{thm}
    Let a function \(\phi\) be closed and convex on a bounded set \(S \subset \R^m\). Consider a linear operator
    \[
        \mathscr{A}(x) = Ax + b: \quad \R^n \to \R^m  
    \]
    Then the function \(f(\xB) = \phi(\mathscr{A}(x))\) is closed and convex on the inverse image of the set \(S\) as follows:
    \[
        Q = \{ \xB \in \R^n \mid \mathscr{A}(\xB) \in S \}    
    \]
\end{thm}

\subsection{Continuity and Differentiability}\label{subsec:Continuity-and-Differentiability}
\begin{thm}
    Let \(f\) be convex and \(\xB_0 \in \mathrm{int}(\dom f)\). Then \(f\) is locally bounded
    and locally Lipschitz continuous at \(\xB_0\).
\end{thm}

\begin{defn}
    Let \(\xB \in \dom f\). We call \(f\) \emph{differentiable} at point \(\xB\) in direction \(\pB \ne 0\)
    if the following limit exists:
    \begin{equation}
        f'(\xB; \pB) = \lim_{\alpha \to 0} \frac{1}{\alpha} [f(\xB + \alpha \pB) - f(\xB)].
    \end{equation}
    The value \(f'(\xB; \pB)\) is called the \emph{directional derivative} of \(f\) at \(\xB\).
\end{defn}

\begin{thm}
    A convex function \(f\) is differentiable in any direction at any
interior point of its domain.
\end{thm}

\begin{lemma}
    \begin{equation}
        f(\yB) \ge f(\xB) + f'(\xB; \yB - \xB).
    \end{equation}
\end{lemma}

\subsection{Separation Theorems}\label{subsec:sep-thm}
\begin{defn}
    Let $Q$ be a convex set. We say that the hyperplane
    \[
        \mathscr{H}(\gB, \gamma) = \{ \xB \in \R^n \mid \innerprod{\gB}{\xB} = \gamma\},\quad \gB \ne 0,  
    \]
    is supporting to \(Q\) if any \(\xB \in Q\) satisfies inequality \(\innerprod{\gB}{\xB} \le \gamma\). 
    The hyperplane \(\mathscr{H}(\gB, \gamma)\) separates a point \(\xB_0\) from \(Q\) if 
    \begin{equation}\label{eq:hyperplane-sep}
        \innerprod{\gB}{\xB} \le \gamma \le \innerprod{\gB}{\xB_0}
    \end{equation}
        
    for all \(\xB \in Q\). If one of the inequalities in \ref{eq:hyperplane-sep} is strict, the we call the separation
    strong.
\end{defn}

\subsection{Subgradients}