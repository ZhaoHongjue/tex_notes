\chapter{Functions and Models}\label{chap:chap1_FuncsModels}

%% Section 1
\section{Function Basics}\label{sec:chap1_func_basics}

\subsection{Functions}\label{subsec:funcs}
\begin{defn}[Function]\label{defn:function}
    A function $f$ is the rule that assigns to each element
    in a set $D$ exactly one element, called $f(x)$, in a set $E$.
\end{defn}

\begin{boxnote}{Notes about Definition~\ref{defn:function}}
    \begin{itemize}
        \item We usually consider functions for which sets $D$ and $E$ are sets of \emph{real numbers}.
        \item The set $D$ is called \textbf{domain} of function $f$.
        \item \textbf{Domain Convention}: The domain of the function is the set of \emph{all inputs for which the formula makes sense and gives a real-number output}.
        \item The number of $f(x)$ is the \textbf{value of $f$ at $x$} and is read $f$ of $x$.
        \item The \textbf{range} of $f$ is the set of all possible values of $f(x)$ as $x$ varies throughout the domain.
        \item A symbol that represents an arbitrary number in the domain of a function $f$ is called an \textbf{independent variable}.
        \item A symbol that represents a number in the range of $f$ is called dependent variable.
        \item 4 ways to represent a function: by an equation, in a table, by a graph, or in words. 
    \end{itemize}
\end{boxnote} 

\begin{thm}[The Vertical Line Test]
    A curve in the $xy$-plane is the graph of a function of $x$ if
    and only if no vertical line intersects the curve more than once.
\end{thm}

If each vertical line at $x = a$ intersects a curve only once, then exactly one function
value is defined by $f(a) = b$. But if a line $x = a$ intersects the curve twice, at $(a, b)$
and $(a, c)$, then the curve can't represent a function because \emph{a function can't assign 2 
different values to $a$} according to Definition~\ref{defn:function}.

\subsection{Piecewise Defined Function}\label{subsec:piecewise_func}
\begin{defn}[Piecewise Defined Function]\label{defn:piecewise_func}
    a function $f$ is called \textbf{piecewise defined function} if the function is defined
    by different formulas in different parts of their domains.
\end{defn}

\begin{example} A function $f$ is defined by
    \[
        f(x) = \begin{cases}
            1 - x & \text{if}~x \le 1\\
            x^2 & \text{if}~x > 1
        \end{cases}
    \]
\end{example}

\begin{example} Absolute value function:
    \[
        f(x) = \abs{x} = \begin{cases}
            x & \text{if} ~ x \ge 0\\
            -x & \text{if} ~ x < 0
        \end{cases}
    \]
\end{example}

\subsection{Even and Odd Functions}\label{subsec:even_odd_funcs}
\begin{defn}[Even Function]\label{defn:even_func}
    If a function $f$ satisfies $f(-x) = f(x)$ for every number $x$ in its domain, then $f$ 
    is called an \textbf{even function}.
\end{defn}

\begin{example}
    The function $f(x) = x^2$ is even because
    $$
        f(-x) = (-x)^2 = x^2 = f(x).
    $$
\end{example}

\begin{defn}[Odd Function]\label{defn:odd_func}
    If a function $f$ satisfies $f(-x) = -f(x)$ for every number $x$ in its domain, then $f$ 
    is called an \textbf{odd function}.
\end{defn}

\begin{example}
    The function $f(x) = x^3$ is odd because
    $$
        f(-x) = (-x)^3 = -(x^3) = -f(x)
    $$
\end{example}

\subsection{Increasing and Decreasing Functions}
\begin{defn}[Increasing and Decreasing Functions]\label{defn:inc_dec_func}
    A function $f$ is called \textbf{increasing function} on interval $I$ If
    $$
        f(x_1) < f(x_2) \quad \text{whenever} ~ x_1 < x_2 ~ \text{for} ~ \forall x_1, x_2 \in I.
    $$
    It's called \textbf{decreasing function} If
    $$
    f(x_1) > f(x_2) \quad \text{whenever} ~ x_1 < x_2 ~ \text{for} ~ \forall x_1, x_2 \in I.
    $$
\end{defn}

\begin{figure}[!htb]
    \centering
    \begin{tikzpicture}
        \draw[->] (-1.5,0) -- (1.5,0);
        \draw[->] (0,-1.5) -- (0,1.5);
        \draw[domain=-1:1] plot(\x,{\x*\x*2 -1});
    \end{tikzpicture}
    \caption{The figure of $f(x) = 2x^2 - 1$}
    \label{fig:inc_dec_func}
\end{figure}

\begin{example}
    For $f(x) = 2x^2 - 1$ which is shown in Fig~\ref{fig:inc_dec_func},
    it's decreasing in interval $(-\infty, 0)$ and it's increasing in interval $(0, +\infty)$.
\end{example}